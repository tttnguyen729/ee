Irrational numbers like \(\pi\) have been approximated ever since they've been first discovered. Nowadays, mathematicians have become so dependent on technology, that they have taken for granted the ease of which they can approximate irrational numbers. Modern calculators have made it a lot easier to approximate irrational numbers like e and \(\pi\). However, when mathematicians rely on technology to do their calculations, they are limited to the accuracy of said technology. For example, a Ti-83 calculator can only approximate \(\pi\) to 9 decimal places, \(3.141592654\). If a better approximation is necessary, a more accurate calculator would be required to solve the problem. Often overlooked approximation methods can be just as accurate as a calculator, and have the potential benefit of being modified to manipulate accuracy as necessary.

As a young child, I was interested in irrational numbers like \(\pi\) and \(\sqrt{2}\). In my IB mathematics class, I learned that although irrational numbers have never-ending decimals that do not repeat, that they can be approximated using series. I was confounded by the idea that adding an endless amount of numbers could result in a finite number, much less an irrational number. My first meaningful interaction with a toilet was when I pondered about the curvature of a toilet. I hypothesized that there was no way the volume of the toilet could be a whole number, yet humans are able to make them nonetheless. This led me to the investigation of approximating curves using Taylor series and to the research question: How well can Taylor series approximations for \(\pi\) be applied in rates of change of cyclical objects?

A Taylor series is an approximation method that can be used to approximate functions and irrational numbers. According to Ron Larson, ``if a function \(f\) has derivatives of all orders at \(x = c\), then the Taylor series for \(f\) at \(c\) is": (Larson 2017, 613).

It is possible to apply this formula where f(n) is the nth derivative to derive the Taylor series for arctan(x) to approximate \(\pi\).

To demonstrate the utility of approximating with Taylor series, the paper will explore the calculations of the volume and rate of water flowing into a toilet bowl. Firstly, rotating a parabola into a solid of revolution will be used to model the structure of the bowl. In this way, manipulation of the upper bound of integration will allow the volume of this bowl to somewhat resemble that of a curved toilet bowl. It is important to note that a normal toilet bowl's structure is not perfectly parabolic. However, this model will defy standard conventions for simplicity. I predict that using a Taylor series will prove to be very accurate and easy to calculate by hand for approximating \(\pi\). 

Taylor series can approximate transcendental numbers using polynomials. This is revolutionary as numbers like \(e\) and \(\pi\) can be viewed not only as irrational numbers, but also as algebraic functions.

To illustrate the graphical property of a Taylor series, the transcendental function \(e^{x}\) will be approximated.

The Taylor series representation of \(e^{x}\) is  

The series was derived from the general formula of a Taylor series.

``Centering the Taylor series at the point c = 0 results in a Maclaurin series" (Larson 2017, 586). Since the derivative of \(e^{x}\) is \(e^{x}\), \(f^{(n)}(0) = 1\) for all \(n\). The series simplifies to:

To approximate \(e\) using the Taylor series, let \(P(n)\) be a polynomial of nth degree:

Using a first order or linear approximation with the polynomial \(P_{1}(x) = 1 + x\):

Not only can Taylor polynomials be used to approximate irrational numbers graphically, they can also approximate numerically. This can be done by plugging an \(x\) value into the Taylor polynomial. Plugging in \(x = 1\) into the function of \(e^x\) will  approximate \(e^1\) or \(e\). 

Plugging in \(x = 1\) into the first order Taylor polynomial, \(P_{1}(x) = 1 + x\), results in \(P_{1}(1) = 1 + 1 = 2\). Let it be noted that the Ti-83 approximates \(e\) as \(2.718281828\).

Using a second order approximation:


Higher order approximations are more accurate than lower order approximations, in this case the second approximation is more accurate than the first and thus matches the curve of the function \(f(x) = e^{x}\) more accurately. The second order approximation for \(e^{x}\) is \(P_{2}(x) = 1 + x + \dfrac{x^2}{2!}\). Plugging in 1 into the Taylor polynomial \(P_{2}(1) = 1 + 1 + 0.5 = 2.5\). As shown by the graph and numerical value of error, the function approximation gets closer to the value of e as more terms are added.

Using a third order approximation of \(P_{3}(x) = 1 + x + \dfrac{x^{2}}{2!} + \dfrac{x^{3}}{3!}\):

An 8 term approximation is strikingly similar to the graph of \(y = e^{x}\)

An eighth order approximation for is \(P_{8}(x) = 1 + x + \dfrac{x^2}{2!} + \dfrac{x^3}{3!} + \dfrac{x^4}{4!} + \dfrac{x^5}{5!} + \dfrac{x^6}{6!} + \dfrac{x^7}{7!} + \dfrac{x^8}{8!}\)

Plugging in 1 into the 8th degree Taylor polynomial:

This value is substantially closer to the value of e than the previous approximations. The two functions are almost indistinguishable on the domain \([0,5]\). As more terms are included in the approximation, the approximation begins to resemble the function approximated more accurately. ``In fact, a Taylor series with infinite terms would converge to the exact value of \(\pi\)" (Jerison 2006). In this way, Taylor series could prove to be a useful way to approximate \(\pi\). As such, I predict that the Taylor series derived will be more accurate than the approximation of the Ti-83 calculator in accuracy with a reasonable number of terms.

Since \(tan(\frac{\pi}{4}) = 1\), then \(arctan(1)= \frac{\pi}{4}\). It is now possible to plug in \(x = 1\) into the arctan equation, with the idea that \(arctan(1) = \frac{\pi}{4}\). It is then possible to multiply this approximation by 4 to get an approximation for \(\pi\).



The first derivative of arctan(x) is 

According to Jerison, the derivative of \(y = arctan(x)\) can be derived with the following triangle using basic trigonometry and calculus (Jerison 2006).

This triangle allows the derivation of:

Solving for y results in arctan

To calculate the derivative of tan(y), a trigonometric identity is used to first define tan(y) in terms of sin(y) and cos(y).

Applying the differential operator on both sides.

Taking the derivative of both sides

Applying the quotient rule:

Simplifying by combining like terms

Although it is possible to use Pythagorean's identity to further simplify \(cos^{2}(y) + sin^{2}(y) = 1\) to get \(\dfrac{1}{cos^{2}(y)} = sec^{2}(y)\), simplifying by splitting into two fractions will be used instead. 

This does not seem useful as the derivative of \(tan(y)\) is defined in terms of \(tan(y)\), however \(y\) was also previously defined as \(arctan(x)\), which allows deriving the derivative of \(y = arctan(x)\) by implicitly differentiating \(tan(y) = x\).

Chain Rule results in dy/dx

Solving for dy/dx

Substituting \(tan(y)\) for \(x\) to find the derivative of \(y = arctan(x)\).

Calculating the second derivative using Power Rule.

The 3rd derivative of \(y = arctan(x)\) is:

Even though plugging in \(x = 0\) to even termed derivatives results in 0, the derivatives become more and more complex with higher and higher derivatives. Using Taylor's formula to make a Taylor series for \(y = arctan(x)\) becomes less efficient as the order of approximation increases and thus another method to derive a Taylor series for \(y = arctan(x)\) will be needed.

Starting with:

Taking the derivative:

The derivative has the form of the following function where \(u = x^2\):

According to Larson, "geometric series converge to the following sum" (Larson 2017, 545).

Manipulating the series by substituting \(-u\) for \(r\) where \(u = r\) and \(a = 1\):

Substituting \(x^{2}\) for \(u\): 

Therefore \(f'(x)\) can be represented by the series:

To derive the series for \(f(x) = arctan(x)\), integrate the series for \(f'(x)\):

\(arctan(0) = 0\) which means \(C = 0\):

The Taylor series constructed previously for \(arctan(x)\) is:

It is convenient that the function \(arctan(x)\) evaluated at 1 is \(\dfrac{\pi}{4}\).

Substituting \(x\) for 1.

Writing out the first 4 terms of this series, it is shown that this is an alternating series; the \((-1)^{n}\) causes the terms to alternate from positive to negative.

Solving for \(\pi\).

According to Loistl, ``Taylor series may not be accurate or even be erroneous if the series diverges" (Loistl 1976). Christine Palmer claims that, ``In practice the Taylor series does converge to the function for most functions of interest, so that the Taylor series for a function is an excellent way to work that function" (Palmer 1998). Even so, as per Loistl, it is necessary to prove that the Taylor series representation of \(arctan(x)\) is convergent by using the Alternating Series Test. If the series is not convergent at \(x = 1\), then the series will not be a good approximation for \(\pi\).

The alternating series test will be used to test if the series for arctan(x) is convergent. According to Ron Larson, an alternating series converges when the two conditions listed below are met (Larson 2017, 567).

First, the limit of the nth term must equal zero.

And the function must be decreasing.

Calculating the limit of the nth term will test whether the series meets the first requirement.

The limit of the nth term as n approaches infinity is zero as the denominator grows increasingly large while the numerator remains constant.

\(a_{n+1} \leq a_{n}\), therefore the function is decreasing for all n. 

Therefore the series converges by the Alternating Series Test.

The definition of absolute convergence will determine whether the series derived for the function arctan(x) is absolutely convergent.

According to Ron Larson, absolute convergence is defined as: 

``The series \(\sum a_{n}\) is absolutely convergent when \(\sum \abs{a_{n}}\) converges" (Larson 2017, 570).

The series derived for arctan(x):

Taking the absolute value of the series.

In other words, it is necessary to test if the non alternating part is convergent.

This behaves like the harmonic series \(\sum\limits_{n=0}^{\infty} \dfrac{1}{n}\) which is divergent. Therefore the alternating series for arctan(x) is not absolutely convergent. However since it is conditionally convergent, it is necessary to check the interval convergence to determine if the series can be used to approximate \(\pi\) at \(x = 1\) by checking if \(x = 1\) falls within the interval in which the series representation of arctan(x) converges.

The ratio test will be used to determine the interval of convergence of a series. For a series to be convergent, the sequence of its partial sum is also convergent. ``Convergent sequences are sequences whose terms approach limiting values" (Strang 2016, 533). The interval of convergence is ``the set of real numbers x for which a power series converges" (Strang 2016, 6). The interval of convergence helps determine if the series is convergent at \(x = 1\) based on whether or not \(x = 1\) is inside the interval.

The series representation of \(arctan(x)\):
 
The Ratio Test is a limit test of the ratio between the n + 1 term and the nth term which allows testing for convergence. ``If \(L < 1\), then the series converges absolutely. If \(L > 1\), then the series is divergent and if \(L = 1\), then the test is inconclusive" (Larson 2017, 575).

Plugging in the nth + 1 term and then multiplying by the reciprocal of the nth term.

As there are absolute value bars, it is possible to ignore the alternating parts because they will not affect the value of the function.

Simplifying and taking \(x^{2}\) out of the limit,


As shown previously, the series is convergent within \(-1 < x < 1\), however it is still necessary to check the bounds for convergence. If the series proves to not converge at \(x = 1\), then the series is not a good predictor of \(\pi\) as the approximation relies on the fact that \(arctan(1) = \pi/4\).


Plugging in \(x = -1\),

Combining the alternating parts results in:

Plugging in \(x = 1\).

Simplifying,

These both pass the alternating series test as the limit of their nth term is zero and the series is always decreasing. Therefore the interval of convergence for \(arctan(x)\) is \( -1 \leq x \leq 1 \). This means that the Taylor series for \(arctan(x)\) can be approximated at x=1 to approximate \(\pi/4\).

The more accurate an approximation needs to be, the more terms will be needed in the Taylor polynomial. However, this is a more difficult task than just plugging in values into the derived Taylor polynomial because the goal is to figure out just how accurate or long the Taylor polynomial needs to be to reach a certain level of accuracy.

According to Christine Palmer, ``\(f(x) = T_{n}(x) + R_{n}(x)\)" where \(T_{n}(x)\) is the approximation and \(R_{n}(x)\) is error (Palmer 1998). However, solving for \(R_{n}(x)\) directly would not be beneficial because \(f(x)\) is irrational and would result in another approximation.
According to Calculus for AP, ``If a convergent alternating series satisfies the condition \(a_{n+1} \leq a_{n} \), then the absolute value of the remainder \(R_{n}\) involved in approximating the sum \(S\) by \(S_{n}\) is less than (or equal to) the first neglected term" (Larson 2017, 569).

This concept allows us to figure out how many terms are needed to beat the Ti-83 calculator's approximation.

A Taylor approximation for \(\pi\) where \(n\) is the index and \(n + 1\) is how many terms the approximation uses:

Plugging in \(x = 1\)

The approximation using only one term is:


It is possible to find the error by taking the absolute value of  the subtraction of the actual value of \(\pi\) by the approximation, in this case \(P(0) = 4\). However, this is problematic as \(\pi\) is an irrational number that is being approximated in the first place. Instead, the error of the approximation can be found by using the Alternating Series Remainder.

As noted earlier, the Alternating Series Remainder is the value of the next unused term. The error of the first approximation using the next unused term is:

To continue, a two term approximation of \(\pi\) is:


And this two term approximation has an error of:


A five term approximation of \(\pi\) is:


A five term approximation has an error of: 


As seen prior, the Taylor series gets more accurate the more terms it uses. It is important to note that the approximation can be either an over approximation or under approximation because the approximation is an alternating series. According to Lowry-Duda, ``the graph of the original function subtracted by the Taylor polynomial can be used as a graphical representation of error" (Lowry-Duda 2019). This is another way to visually see the difference between the ninth order and two term approximation of \(arctan(x)\). 



The graph of the error of the ninth order approximation is closer to zero on a larger domain than the graph of the error of the two term approximation. This is further evidence that a Taylor series gets more accurate as more terms are added.

The Ti-83's approximation for \(\pi\) is: \(3.141592654\). This is accurate to 9 decimal places or \(10^{-9}\)

In order to calculate the least amount of terms that are needed to beat the Ti-83's approximation, I had to rely on a Python program to continually find the next unused term that would be less than or equal to \(10^{-9}\). See appendix for the Python program.

The output of the program is:
A Taylor series approximation would need almost 2 billion terms to beat the Ti-83's approximation of \(\pi\), rendering it impossible to calculate using the Taylor series at this level of accuracy. In this way, although a Taylor series approximation of \(\pi\) can substitute technology for basic calculations, it is not as effective as basic technology like a Ti-83 calculator. The difficulty in using Taylor series is not only determining how many terms are needed but adding their partial sum. Since the Taylor series converges so slowly, more terms are required to reach a certain level of accuracy. Without the aid of technology, a Taylor series would only be feasible to use with a limited number of terms.

The Wolfram Alpha approximation for \(\pi\) is accurate to 57 decimal places: \\
\(3.141592653589793238462643383279502884197169399375105820974\)

It is difficult to even determine how many terms in the Taylor series is necessary to overcome this level of accuracy. As such, the Taylor series approximation derived can not reasonably compare to Wolfram Alpha's approximation in terms of accuracy.

A Taylor series with 2 terms has an error of less than or equal to 0.8.

Expanding the Taylor approximation to 12 terms reduces the error to less than or equal to 0.16. This is about as effective as approximating \(\pi\) as 3 which has an error around .1415926536.

To get an approximation better than 3.14 or have an error less than or equal to 0.01, 199 terms are needed. 

A toilet bowl is being filled with water at a constant rate. At what rate is the level of water rising when the water in the tank is \(2 \; cm\) deep?


As noted earlier, flushing the toilet causes the rate of water flowing into the toilet to spike initially, then gradually slowing down, until no more water flows into the toilet. However, for the sake of the related rates problem, a constant flush rate will be assumed as the priority is to apply the approximation to a real life problem. It is also important to note that the volume of the toilet is not completely filled with water.  

The graph of \(x^{2}\) will be used to model the curvature of the toilet and the cross section of the toilet bowl. More specifically, the solid of revolution around the y axis will create a toilet shaped bowl. 


Integrating area to get volume through the disc method.
Substituting the radius for \(\sqrt{y}\).

Applying the first fundamental theorem of Calculus.

In order to find the rate at which the volume of water flows into the toilet, the volume of a toilet as well as how long it takes the toilet to flush will need to be calculated. 

The toilet in my school's bathroom uses \(13.2 \; L\) per flush. With this information, another assumption is that after the toilet finishes flushing, it will hold \(13.2 \, L\) of water. This oversimplification of the problem is just to illustrate the use of a Taylor series. Using a stopwatch, I averaged how long it takes for twenty toilet flushes. The timer started as soon as the toilet bowl emptied and stopped when water stopped filling the bowl to measure the rate at which the height of water is rising.


The average time for these 20 flushes is 15.50 seconds. 

To keep units between height and volume similar, volume in liters will be converted to cubic centimeters.

To find \(\dfrac{dV}{dt}\), the rate at which volume changes, divide volume by time. In this case,

It is true that normal toilets do not flush at a constant rate. Nonetheless,  a constant rate of 851.613 \(\dfrac{cm^{3}}{sec}\) that was derived from averaging the time it takes a toilet to flush will be assumed. The related rates problem becomes: A toilet bowl is being filled with water at a constant rate of \(851.613 \, cm\). At what rate is the level of water rising when the water in the tank is \(2 \; cm\) deep?

The formula for the volume of the toilet is:

Differentiating volume with respect to time where dy/dt represents the rate of change of the height of water in the toilet bowl.

199 terms will be needed in the Taylor series approximation to beat an approximation of using 3.14.


Substituting in the approximation of \(\pi\) and simplifying.


Solving for dy/dt.

The rate the water's height is increasing at the instant \(y = 2 \; cm\) will be found.

Simplifying

Plugging the equation into a calculator produces the result:


The Taylor series approximation is very inaccurate with few terms but gradually gets more accurate as more terms are used. Even termed approximations are underapproximations while odd termed approximations are overapproximations which is expected of an alternating series. Even at 1000 terms, the Taylor series still has an error of over 0.01, demonstrating the series' slow rate of convergence. It is interesting that the approximation error with respect to the Ti-83 approximation gets smaller at nearly the same rate that the number of terms increases. Approximating \(\pi\) as 3 results in a calculation of 45.17947285 which has a difference from the Ti-93's approximation of -2.13236048. In this way, the Taylor series with 10 terms is a better approximation for \(\pi\) than 3. 

According to Ron Larson, ``this series (developed by James Gregory in 1671) is not a practical way of approximating \(\pi\) because it converges so slowly that hundreds of terms would have to be used to obtain reasonable accuracy" (Larson 2017, 609). In fact, it was derived that 199 terms are required to be as accurate as 3.14 and requires almost two billion terms to even come close to beating a Ti-83's accuracy. This is just not feasible to do without the aid of technology. Moreover, it was necessary to use technology to quantify how many terms would be needed to beat the Ti-83's approximation.

Although the Taylor series found is not practical for approximating \(\pi\) for everyday use. There are lots of other series that can be used to approximate \(\pi\). 

According to Weisstein, ``there are series that even converge directly to \(\pi\)" (Weisstein 2002).


These function functions, although more accurate than the Gregory-Leibniz series used in the research investigation, were not used due to their complex nature. The first series requires knowledge on the Riemann zeta function and the second was not used due to the complexity of its derivation.







